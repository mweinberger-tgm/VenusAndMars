%%%%%%%%%%%%%%%%%%%%%%%%%%%%%%%%%%%%%%%%%%%%%%%%%%%%%%%%%%%%%%%%%
% Document Class
%%%%%%%%%%%%%%%%%%%%%%%%%%%%%%%%%%%%%%%%%%%%%%%%%%%%%%%%%%%%%%%%%
\documentclass[12pt,a4paper,oneside,ngerman]{scrartcl}


%%%%%%%%%%%%%%%%%%%%%%%%%%%%%%%%%%%%%%%%%%%%%%%%%%%%%%%%%%%%%%%%%
% Packages
%%%%%%%%%%%%%%%%%%%%%%%%%%%%%%%%%%%%%%%%%%%%%%%%%%%%%%%%%%%%%%%%%
\usepackage[ngerman]{babel}
\usepackage[utf8]{inputenc}
\usepackage[nottoc,numbib]{tocbibind}
\usepackage[T1]{fontenc}
\usepackage{fancyhdr}
\usepackage{booktabs}
\usepackage[page]{totalcount}
\usepackage{tikz}
\usepackage{color, colortbl}
\usepackage{url}
\usepackage{listings}
\usepackage{tabularx}
\renewcommand{\familydefault}{\sfdefault}
\usepackage{lmodern}
\usepackage{geometry}
\usepackage{ragged2e}
\usepackage{ulem}


%%%%%%%%%%%%%%%%%%%%%%%%%%%%%%%%%%%%%%%%%%%%%%%%%%%%%%%%%%%%%%%%%
% Colors
%%%%%%%%%%%%%%%%%%%%%%%%%%%%%%%%%%%%%%%%%%%%%%%%%%%%%%%%%%%%%%%%%
\definecolor{g1}{RGB}{224,255,255}
\definecolor{g2}{RGB}{37,147,37}
\definecolor{g3}{RGB}{29,114,29}
\definecolor{g4}{RGB}{20,42,20}
\definecolor{listings}{rgb}{0.96, 0.96, 0.96}

% Java Syntaxhighligthning
% strings
\definecolor{javared}{rgb}{0.6,0,0}
% comments
\definecolor{javagreen}{rgb}{0.25,0.5,0.35}
% keywords
\definecolor{javapurple}{rgb}{0.5,0,0.35}
% javadoc
\definecolor{javadocblue}{rgb}{0.25,0.35,0.75}


%%%%%%%%%%%%%%%%%%%%%%%%%%%%%%%%%%%%%%%%%%%%%%%%%%%%%%%%%%%%%%%%%
% Page Margin
%%%%%%%%%%%%%%%%%%%%%%%%%%%%%%%%%%%%%%%%%%%%%%%%%%%%%%%%%%%%%%%%%
\geometry{
  left=2.5cm,
  right=2.5cm,
  top=2.5cm,
  bottom=2.5cm
}


%%%%%%%%%%%%%%%%%%%%%%%%%%%%%%%%%%%%%%%%%%%%%%%%%%%%%%%%%%%%%%%%%
% Definitions
%%%%%%%%%%%%%%%%%%%%%%%%%%%%%%%%%%%%%%%%%%%%%%%%%%%%%%%%%%%%%%%%%\\

\lstdefinestyle{Java}{
	language=Java,
	keywordstyle=\color{javapurple}\bfseries,
	stringstyle=\color{javared},
	commentstyle=\color{javagreen},
	morecomment=[s][\color{javadocblue}]{/**}{*/},
}

%%%%%%%%%%%%%%%%%%%%%%%%%%%%%%%%%%%%%%%%%%%%%%%%%%%%%%%%%%%%%%%%%
% Own Commands
%%%%%%%%%%%%%%%%%%%%%%%%%%%%%%%%%%%%%%%%%%%%%%%%%%%%%%%%%%%%%%%%%
\raggedright
\newcommand{\tabhvent}[1]{\noindent\parbox[c]{\hsize}{#1}}
\newcolumntype{b}{X}
\newcolumntype{s}{>{\hsize=.5\hsize}X}


%%%%%%%%%%%%%%%%%%%%%%%%%%%%%%%%%%%%%%%%%%%%%%%%%%%%%%%%%%%%%%%%%
% Document Start
%%%%%%%%%%%%%%%%%%%%%%%%%%%%%%%%%%%%%%%%%%%%%%%%%%%%%%%%%%%%%%%%%
%%%%%%%%%%%%%%%%%%%%%%%%%%%%%%%%%%%%%%%%%%%%%%%%%%%%%%%%%%%%%%%%%
% Title Page
%%%%%%%%%%%%%%%%%%%%%%%%%%%%%%%%%%%%%%%%%%%%%%%%%%%%%%%%%%%%%%%%%
\begin{document}
\thispagestyle{empty}
\vspace*{2cm}


\begin{center}
\begin{huge}
\renewcommand{\ULthickness}{2pt}
\uline{Dokumentation}
\\
\uline{Venus and Mars}
\end{huge}
\end{center}

\vspace{9cm}

\textit{\textbf{Teammitglieder: Thomas Taschner, Michael Weinberger}}
\vspace{10mm}

\textbf{{\color{g4}Version 0.2 \hfill 23.11.2015 \hfill Status: [DRAFT]}}
\\
%Table
\begin{table}[h]
\renewcommand{\arraystretch}{2.1}
\centering
\begin{tabularx}{\textwidth}{|s|s|b|b|}

\specialrule{0.07em}{0em}{0em}
\multicolumn{4}{|l|}{\textbf{Git-Pfad:} /doc/ \hfill \textbf{Dokument:} dokumentation.tex} \\ \hline
\end{tabularx}
\end{table}
\newpage


%%%%%%%%%%%%%%%%%%%%%%%%%%%%%%%%%%%%%%%%%%%%%%%%%%%%%%%%%%%%%%%%%
% Header & Footer
%%%%%%%%%%%%%%%%%%%%%%%%%%%%%%%%%%%%%%%%%%%%%%%%%%%%%%%%%%%%%%%%%
\pagestyle{fancy}
\renewcommand{\headrulewidth}{0.4pt}
\renewcommand{\footrulewidth}{0.4pt}
\setlength\headheight{15pt}
\lhead{Solarsystem}
\rhead{Version 0.2}
\lfoot{Taschner, Weinberger}
\cfoot{}
\rfoot{Seite \thepage \hspace{1pt} von \totalpages}


%%%%%%%%%%%%%%%%%%%%%%%%%%%%%%%%%%%%%%%%%%%%%%%%%%%%%%%%%%%%%%%%%
% Table of Contents
%%%%%%%%%%%%%%%%%%%%%%%%%%%%%%%%%%%%%%%%%%%%%%%%%%%%%%%%%%%%%%%%%
\tableofcontents\thispagestyle{fancy}
\newpage


%%%%%%%%%%%%%%%%%%%%%%%%%%%%%%%%%%%%%%%%%%%%%%%%%%%%%%%%%%%%%%%%%
% Changelog
%%%%%%%%%%%%%%%%%%%%%%%%%%%%%%%%%%%%%%%%%%%%%%%%%%%%%%%%%%%%%%%%%
\section{Changelog}

%Table
\begin{table}[h]
\renewcommand{\arraystretch}{3.0}
\centering
\begin{tabularx}{\textwidth}{|s|s|s|s|b|}
\hline
\rowcolor{g1} 

%Header
\tabhvent{\textbf{Version}} & \tabhvent{\textbf{Datum}} & \tabhvent{\textbf{Status}} & \tabhvent{\textbf{Bearbeiter}} & \tabhvent{\textbf{Kommentar}}  \\ \hline

%Lines
\tabhvent{\textbf{0.1}} & \tabhvent{14.10.2015} & \tabhvent{Erstellt} &  \tabhvent{Thomas Taschner} & \tabhvent{Erstellt und Inhalte hinzugefügt}         \\
\hline
\tabhvent{\textbf{0.2}} & \tabhvent{14.10.2015} & \tabhvent{Erstellt} &  \tabhvent{Michael Weinberger} & \tabhvent{Erstellt und Inhalte hinzugefügt}         \\
\hline


\end{tabularx}
\end{table}
\newpage


%%%%%%%%%%%%%%%%%%%%%%%%%%%%%%%%%%%%%%%%%%%%%%%%%%%%%%%%%%%%%%%%%
% Content Start
%%%%%%%%%%%%%%%%%%%%%%%%%%%%%%%%%%%%%%%%%%%%%%%%%%%%%%%%%%%%%%%%%
\justify
%%%%%%%%%%%%%%%%%%%%%%%%%%%%%%%%%%%%%%%%%%%%%%%%%%%%%%%%%%%%%%%%%
% Übersicht
%%%%%%%%%%%%%%%%%%%%%%%%%%%%%%%%%%%%%%%%%%%%%%%%%%%%%%%%%%%%%%%%%
\section{Übersicht}
Erstelle eine einfache Animation unseres Sonnensystems!

\section{Projektbeschreibung}
\subsection{Anforderungen}
In einem Team (2 Personen) sind folgende Anforderungen zu erfüllen.

\begin{itemize}
\item Ein zentraler Stern
\item Zumindest 2 Planeten, die sich um die eigene Achse und in elliptischen Bahnen um den Zentralstern drehen
\item Ein Planet hat zumindest einen Mond, der sich zusätzlich um seinen Planeten bewegt
\item Weitere Planeten, Asteroiden, Galaxien,...
\item Zumindest ein Planet wird mit einer Textur belegt
\end{itemize}

Events:

\begin{itemize}
\item Mittels Maus kann die Kameraposition angepasst werden: Zumindest eine Überkopf-Sicht und parallel der Planentenbahnen
\item Da es sich um eine Animation handelt, kann diese auch gestoppt werden.
\item Mittels Tasten kann die Geschwindigkeit gedrosselt und beschleunigt werden.
\item Mittels Mausklick kann eine Punktlichtquelle und die Textierung ein- und ausgeschaltet werden.
\item Auch Monde und Planeten werfen Schatten.
\end{itemize}

Hinweise zu OpenGL und glut:

\begin{itemize}
\item Ein Objekt kann einfach mittels glutSolidSphere() erstellt werden.
\item Die Planten werden mittels Modelkommandos bewegt: glRotate(), glTranslate().
\item Die Kameraposition wird mittels gluLookAt() gesetzt.
\item Entfernte Objekte sollen kleiner und nahe Objekte größer.
\item Wichtig ist dabei auch eine möglichst glaubhafte Darstellung. \newline gluPerspective(), glFrustum()
\item Für das Einbetten einer Textur kann die Library Pillow verwendet werden!
\end{itemize}
\newpage
\subsection{Teammitglieder}
\begin{table}[h]
\renewcommand{\arraystretch}{3.0}
\centering
\begin{tabularx}{\textwidth}{|s|s|s|s|b|}
\hline


\tabhvent{\textbf{Name}} &\tabhvent{\textbf{Rolle}}  \\ \hline

\tabhvent{Thomas Taschner, Michael Weinberger} & \tabhvent{Entwickler, Dokumentation} \\ \hline


\end{tabularx}
\end{table}

\newpage
\section{Evaluierung der Frameworks }
\subsection{Pygame}


%%%%%%%%%%%%%%%%%%%%%%%%%%%%%%%%%%%%%%%%%%%%%%%%%%%%%%%%%%%%%%%%%
% Bibliography
%%%%%%%%%%%%%%%%%%%%%%%%%%%%%%%%%%%%%%%%%%%%%%%%%%%%%%%%%%%%%%%%%
\bibliography{Dokumentenvorlage} 
\bibliographystyle{ieeetr}


%%%%%%%%%%%%%%%%%%%%%%%%%%%%%%%%%%%%%%%%%%%%%%%%%%%%%%%%%%%%%%%%%
% Content End
%%%%%%%%%%%%%%%%%%%%%%%%%%%%%%%%%%%%%%%%%%%%%%%%%%%%%%%%%%%%%%%%%
\end{document}
%%%%%%%%%%%%%%%%%%%%%%%%%%%%%%%%%%%%%%%%%%%%%%%%%%%%%%%%%%%%%%%%%
% Document End
%%%%%%%%%%%%%%%%%%%%%%%%%%%%%%%%%%%%%%%%%%%%%%%%%%%%%%%%%%%%%%%%%